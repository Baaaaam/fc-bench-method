\section{Introduction}
\label{intro}
The act of fuel cycle benchmarking has long faced methodological issues 
on what metrics to compare, how to compare them, and at what point in the
fuel cycle they should be compared. This is partly because such activities 
are not benchmarking in the strictess validation sense. Most fuel
cycle benchmarks are more correctly called code-to-code comparisons or 
inter-code comparisons, as they compare simultor results. Importantly, 
these take place in the absence of true experimental data. The number of 
real-world, industrial scale nuclear fuel cycles that have historically been 
deployed is not sufficient for statistical accuracy even for the Once-Through 
sceanario. For other fuel cycles, industrial data is even more stark. 
Since fuel cycle simulation is thus effectively impossible to validate, 
we should look to methods non-judgemental methods of benchmarking. The 
results of any given simulator should be evaluated in reference to how 
it performs against other simulators in such a way that acknowledges that 
any and all simulators may demonstrate incorrect behaviour. No simulator
by fiat produces the `true' or reference case.

The other major conceptual issue with fuel cycle benchmarking is that there 
is no agreed upon mechanism for establishing a figure-of-merit (FOM) for 
a metric that is uniform across all metrics of interest. For example, 
repository heat load may be examined only at the end of the of the simulation,
separatred plutonium may be used whereever it peaks, and natural uranium 
mined might be of concern only in 100 years. Comparing at a specific point 
in time fails to take into account the behaviour of that metric over time and 
can skew any decisons made based soley on that metric. Additionally, the 
time of comparison varies based on the metric itself. This is a necessary 
side effect of picking a single point in time.
Furthermore, such FOMs are not useful for indicating why simulations differ, 
only that they do. Moroever, if such FOM match, this does not indicate
that the simulator actually agree. Their behaviour could be radically 
different at every other point in time.  It should be noted that 
Equillibrium and quasi-static fuel cycle simulators are sometimes able to 
ignore these issues, because all time points are treated equally.

Some dynamic FOMs do exist. 



Such as cyclus \cite{cyclus_v1_2}.