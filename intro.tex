\section{Introduction}
\label{intro}
The act of fuel cycle benchmarking has long faced methodological issues 
on what metrics to compare, how to compare them, and at what point in the
fuel cycle they should be compared. The benchmarking mechanism described 
here couples Gaussian process models (GP) to dynamic time warping (DTW).
Together these address how to perform fundemental tasks for common nuclear
fuel cycle benchmarking. 

Confusion in this area is partly because such activities 
are not benchmarking in the strictess validation sense. Most fuel
cycle benchmarks are more correctly called code-to-code comparisons or 
inter-code comparisons, as they compare simultor results. Importantly, 
these take place in the absence of true experimental data. The number of 
real-world, industrial scale nuclear fuel cycles that have historically been 
deployed is not sufficient for statistical accuracy even for the Once-Through 
sceanario. For other fuel cycles, industrial data is even more stark. 
Since fuel cycle simulation is thus effectively impossible to validate, 
we should look to methods non-judgemental methods of benchmarking. The 
results of any given simulator should be evaluated in reference to how 
it performs against other simulators in such a way that acknowledges that 
any and all simulators may demonstrate incorrect behaviour. No simulator
by fiat produces the `true' or reference case.

The other major conceptual issue with fuel cycle benchmarking is that there 
is no agreed upon mechanism for establishing a figure-of-merit (FOM) for 
a metric that is uniform across all metrics of interest. For example, 
repository heat load may be examined only at the end of the of the simulation,
separatred plutonium may be used whereever it peaks, and natural uranium 
mined might be of concern only in 100 years. Comparing at a specific point 
in time fails to take into account the behaviour of that metric over time and 
can skew any decisons made based soley on that metric. Additionally, the 
time of comparison varies based on the metric itself. This is a necessary 
side effect of picking a single point in time.
Furthermore, such FOMs are not useful for indicating why simulations differ, 
only that they do. Moroever, if such FOM match, this does not indicate
that the simulator actually agree. Their behaviour could be radically 
different at every other point in time.  It should be noted that 
Equillibrium and quasi-static fuel cycle simulators are sometimes able to 
ignore these issues, because all time points are treated equally.

Some dynamic FOMs do exist. However, these typically require that the metric
data be too well-behaved for comparison purposes. Consider the case of total
power produced [GWe]. A FOM could be the sum over time of the relative error 
between the total power of a single simulator and the root-mean squared total power
of all the simulators together. However, such an FOM fails if the total power
time series have different lengths. Such differences could arrise because 
of different time steps (1 month versus 1 year) or because of different 
simulation durations. Suppose that a comparison is posed as "until transition"
in a transion scenario, it could defeat the purpose of the benchmark to
force different simulators to have the same time-to-transition if they 
nominally would have had distinct. 
 
The mechanisms used for benchmarking that have been discussed so far typically
do not incorporate modeling uncertainy coming from the simulator itself.
This is likely because most simulators do not compute uncertainty on their 
own. Instead they rely on perturbation studies or stochatic wrappers around 
the simulator. Furthermore, metrics may add their own uncertainty from the 
data that the bring in (half-lives, cross-sections, etc.) and the 
operations they perform. However, even if such error bars were available for
every point in a time series metric, the benchmark FOM calculation would 
ignore them.

The method described in this paper addresses all of the above issues that 
in a way that a dynamic fuel cycle benchmark will be able to use on the 
metrics of interest. It is important to note, though, that most fuel cycle 
metrics are time series and can be derived from the mass balances. 
Furthermore, most metrics have an associated total metric that can be 
computed from the linear combination of all of its constituent features. 
For example, total mass flows are the sum of the mass flow of each nuclide
and total power generated is the sum of the power from each reactor type 
(such as LWRs and FRs).  These attributes are common to the overwhelming 
majority of fuel cycle metrics.

GP

DTW

Such as cyclus \cite{cyclus_v1_2}.