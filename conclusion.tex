\section{Conclusions \& Future Work}
\label{conclusion}

This paper demonstrates a robust method for generating figures-of-merit
for nuclear fuel cycle benchmarking activities by coupling Gaussian process
regression to dynamic time warping. This method takes advantage of modeling
uncertainties in fuel cycle metrics if they are known. It is also capable 
of handling the situation where different simulators output metric data on
vastly different time grids. The distance computed by the dynamic time 
warping can itself serve as the figure-of-merit. Additionally, the 
distance can also be used to derive contribution and normalized contribution
figures-of-merit. These measures are valuable for non-judgementally 
determining the impact of different constiuent measures to a total 
metric, e.g. LWR versus FR power generation. The contribution metrics also 
scale such that higher values imply higher impact. This is sometimes
more intuitve as compared to a DTW distance, where lower values imply
more similar curves.

Any regression method could have been used to form a model. Similarly, any
mechanism for comparing two time series could have been used as a measure
of distance.  However, Gaussian processes and DTW were chosen because
fuel cycle simulator, and thus benchmarks and inter-code comparisons, lack 
the ability to be experimentally validated.
It is not possible to build out fully specifcy a fuel cycle scenario,
physically contruct the initial conditions by building actual facilities, 
and then record how the cycle performs at a future date. 
Going through this process for many scenrios and initial conditions that 
a benchmark may examine is likewise impossible in practice.
Furthermore, using 
historical data for validation provides too few cases for comparison and 
each simulator could simply be tuned to precisely match historical events.
Thus, each simulator in a benchmark could be valid or they all could be 
invalid. It is therefore necessary for the FOM to not skew for or against 
any particular simulator. Gaussian process models as used here do not 
judge the simulators differently. The DTW then takes into account the 
cumulative effect of the whole time domain and does not preferentially 
select certain times.

The sample benchmark presented here was very simple and was used for motivation 
purposes only. It consisted of just
two simulators (DYMOND and Cyclus) and one metric (generated power) with
two components (LWR and FR).  However, both Gaussian processes and DTW
are inherently multivariate. In this paper, a univariate formulation of 
Gaussian processes was all that was needed to demonstrtate the method, and 
the $L_1$ norm of DTW can operate on vectors of time series as was seen 
in the dicsussion of MFCCs. Therefore, more complex forms of analysis could 
be performed without modifying the fundemental method. For example, the Gaussian process could jointly model the 
effect from many inputs onto the metric. Perhaps the benchmark is formulated
to look at the generated power as a function of time and the power demand curve.
In this case, a two dimensional GP model would be used. Alternatively, 
suppose that a matrix time series of the all individual nuclide mass flows 
are available. DTW is still able compute the distance between two 
such matrices. This would yield a measure of how the mass flows themselves
differ - taking into account each nuclide component - without relying on a 
collapsed one dimensional total mass flow curve. 
This is valuable as a FOM because it provides a measure of the whole fuel
cycle as a function of each facility, each nuclide, and time 
without needing to explicitly examine the time series for each facility 
and each nulcide.
Such cases will be considered in
future work as real inter-code comparison data becomes available.

Furthermore, this work focused only on benchmarking and inter-code comparison
activities.
However, the FOM calculations presented here could also be used to evaluate 
different fuel cycle scenarios. DTW distances could be computed between
a typical once-through scenario and an LWR-to-FR transition
scenario, or any other proposed scenario. This provides a measure for 
comparing the relative cost (in units of the metric, not necessarily 
economic) for selecting one cycle over another. 
Such measures have the advanatage of being unbiased with respect to the 
simulators that comprise the evaluation study; no reference or best-performing
simulator needs to be chosen. 
Additionally, because DTW may handle different length time series, 
transition scenarios that complete transition at different times can be 
compared directly without the need to run all scenarios out to the time of 
the longest transition.
Furthermore for FOMs that seek to encompass all fuel cycle facilities,
the meausures themselves may need to employ zero-valued `ghost' 
facilities to establish a common basis between scenarios.  For example, 
imaginary FRs with zero material flowing through them may need to be added 
to a once-through scenario to be able to compute a DTW mass balance distance 
to an LWR-to-FR transition scenario. That is, for fuel cycle evaluation,
it is expected that the union of all types of facilities across all scenarios 
will be required.
The work here, thus, 
should be seen as a stepping stone to further fuel cycle scenario evaluation
efforts. It is this upcoming evaluation application is anticipated to more 
directly benefit decision makers with respect to studies such as 
\cite{wigeland2014nuclear} than the benchmark application presented here.
