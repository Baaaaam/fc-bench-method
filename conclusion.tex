\section{Conclusions \& Future Work}
\label{conclusion}

This paper demonstrates a robust method for generating figures-of-merit
for nuclear fuel cycle benchmarking activities by coupling Gaussian process
regression to dynamic time warping. This method takes advantages of modeling
uncertainties in fuel cycle metrics if they are known. It is also capable 
of handling the situation where different simulators output metric data on
vastly different time grids. The distance computed by the dynamic time 
warping can itself serve as the figure-of-merit. Additionally, the 
distance can also be used to derive contribution and normalized contribution
figures-of-merit.

Any regression method could have been used to form a model. Similarly, any
mechanism for comparing two time series could have been used as a measure
of distance.  However, Gaussian processes and DTW were chosen because of 
the nature of a benchmarks and inter-code comparisons that lack experimental
validation. It is impossible to build out a given fuel cycle scenario
and see how it performs 200 years in the future. Furthermore, using 
historical data for validation provides too few cases for comparison and 
each simulator could be tuned simply to match precisely historical events.
Thus, each simulator in a benchmark could be valid or they all could be 
valid. It is therefore necessary for the FOM to not skew for or against 
any particular simulator. Gaussian process models as used here do not 
judge the simulators differently. The DTW then takes into account the 
cumulative effect of the whole time domain and does not preferentially 
select certain times.

The example benchmark presented here was very simple consisting of only
two simulators (DYMOND and Cyclus) and one metric (generated power) with
two components (LWR and FR).  However, both Gaussian processes and DTW
are inherently multivariate. More complex forms of analysis could therefore
be performed. For example, the Gaussian process could jointly model the 
effect from many inputs onto the metric. Perhaps the benchmark is formulated
to look at the generated power as a function of time and the demand curve.
In this case, a two dimensional GP model would be used. Alternatively, 
suppose that a matrix time series of the all individual nuclide mass flows 
are available. DTW is still able compute the distance between two 
such matrices. This would yield a measure of how the mass flows themselves
differ - taking into account each nuclide component - without rely on the
one dimensional total mass flow curve.  Such cases will be considered in
future work as real inter-code comparison data becomes available.

Furthermore, this work focused on the particular use case of benchmarking.
However, the FOM calculations presented here could also be used to evaluate 
different fuel cycle scenarios. DTW distances could be computed for between
a business-as-usual once through scenario and a LWR-to-FR transition
scenario or any other proposed scenario. This provides a measure for 
comparing the relative cost (in units of the metric, not necessarily 
economic) for selecting one cycle over another. The work here, thus, 
should be seen as a stepping stone to further fuel cycle scenario evaluation
work.

Lastly, dynamic time warping could itself serve a purpose as the objective 
function is a fuel cycle optimization.  For example, suppose a demand curve 
such as 1\% power growth is known. The DTW distance from the total generated
power to this curve could be minimized as a function of the reactor 
deployment schedule. Such a distance could potentially yield a more 
precise or faster optimization process than simply taking the sum of 
the differences between two time series. Such an optimization would also allow
for matching on multiple time series features simultaneously while retaining
a real-valued objective function.

