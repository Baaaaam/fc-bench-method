This paper presents a new fuel cycle benchmarking analysis methodology
by coupling Gaussian process regression, a popular technique in Machine 
Learning, to dynamic time warping, a mechanism widly used in speech 
recognition. Together they generate figues-of merit that are applicable to
any time series metric that a benchmark study may use. The figures-of-merit
take account for uncertainty in the metric itself, utilize information
across the whole time domain, and does not require that the simulators
use a common time grid. A distance measure is detailed that can be used to 
compare the perfomance of each simulator for a given metric. Additionally, 
a contribution measure is derived that can be used to rank order metrics
that contribute to a cooresponding total metric. Lastly, this paper 
warns against using standard signal processing techniques for error reduction.
This is because error reduction is better handled by the Gaussian process 
regression itself.