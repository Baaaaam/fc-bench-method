This paper presents a new fuel cycle benchmarking analysis methodology
by coupling Gaussian process regression, a popular technique in Machine 
Learning, to dynamic time warping, a mechanism widely used in speech 
recognition. Together they generate figures-of-merit for a suite of fuel 
cycle realizations. The figures-of-merit may be computed for any time 
series metric that is of interest to a benchmark. For a given metric, 
these figures-of-merit have the advantage that they are reduce the 
dimensionailty to a scalar, and are thus directly comparable. 
The figures-of-merit
account for uncertainty in the metric itself, utilize information
across the whole time domain, and do not require that the simulators
use a common time grid. Here, a distance measure is defined that can be used 
to compare the performance of each simulator for a given metric. Additionally, 
a contribution measure is derived from the distance measure that can be used 
to rank order the impact of different partitions of a fuel cycle metric. 
Lastly, this paper 
warns against using standard signal processing techniques for error reduction,
as error reduction is better handled by the Gaussian process regression 
itself.
