\section{Gaussian Process Modeling}
\label{gp}
The study of Gaussian processes is rich and deep and more completely covered by 
other reseources, such as \cite{rasmussen2006gaussian}. Here only the barest of 
introductions to this topic are given as motivated by the regression problem at
hand. For nuclear fuel cycle benchmark analysis, Gaussian processes will be used
to form a model of the metric time series over all $S$ simulators. 

A Gaussian process is defined by a mean and covariance function. 
The mean function $\mu(t)$ is taken to be the expectation value $\E$ of mean
the input fuctions, which here are the time series for all simulators. The
covariance function $k(t, t^\prime)$ is the expected value of the inputs to the 
mean. Symbolically, 
\begin{equation}
\label{mean-func}
\mu(t) = \E\left[m_i^s(t)\right] = \E\left[m_i^0(t), m_i^1(t), \ldots\right]
\end{equation}
\begin{equation}
\label{covar-func}
k(t, t^\prime) = \E\left[(m_i^s(t) - \mu(t))(m_i^s(t^\prime) - \mu(t^\prime))\right]
\end{equation}
The Gaussian process $\GP$ thus approximates the metric for all simulators. This
is denoted either using functional or operator notation as follows:
\begin{equation}
\label{gp-def-approx}
m_i(t) \approx \GP(\mu(t), k(t, t^\prime)) \equiv \GP m_i^s
\end{equation}

